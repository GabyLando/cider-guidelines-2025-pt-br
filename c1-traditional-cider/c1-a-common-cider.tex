\phantomsection
\subsection*{C1A. Common Cider}
\addcontentsline{toc}{subsection}{C1A. Common Cider}

\textit{A \textbf{Common Cider} is made primarily from culinary (table) apples. Compared to most other styles in this category, these ciders are generally lower in tannin and higher in acidity.}

\textbf{Impressões Gerais}: A refreshing drink with the fruity and floral aroma of apples, and a bright, juicy acidity. Fresh, with a clean fermentation, but possibly showing a slight yeast character.

\textbf{Aroma e Sabor}: Apple character noticeable, either as the flavor of the fruit or as a fruity-floral aroma. Sweet or low-alcohol ciders may have noticeable apple aroma and flavor. Dry ciders will be more neutral-flavored and wine-like with some applederived esters and floral notes. Apple-derived esters are not necessarily apple-like; other fruit notes are possible (similar to what occurs when grapes are fermented into wine). Sweetness and acidity should combine to give a refreshing character. Medium to high acidity adds a refreshing quality, but must not be harsh or biting. Restrained tannin may contribute to an increased perception of dryness in the finish. Generally clean fermentation without the rustic or MLF notes of some other regional ciders. Light yeast character acceptable.

\textbf{Aparência}: Slightly cloudy to brilliant. Color ranges from very pale straw to medium gold. Red-fleshed apple varieties can produce ciders with a blush hue.

\textbf{Sensação na Boca}: Medium-light to medium body. Light tannin can provide a slight to medium-low astringency, but little bitterness. Any level of carbonation.

\textbf{Comentários}: A refreshing drink of some substance - neither bland nor watery. Sweet ciders must not be cloying. Dry ciders must not be too austere (subtle, muted, tight fruit flavor with high acidity). Sometimes called New World Cider or Modern Cider. The name common implies lack of rarity, not lack of quality or class. Common cider may use heirloom apple varieties, if they do not have appreciable tannin levels, significant nonfruity character, or unusual intensity - ciders with these qualities are best entered in other Traditional Cider styles.

\textbf{Instruções para Inscrição}: Entrants \textbf{MUST} specify both carbonation and sweetness levels. Entrants \textbf{MAY} specify apple varieties, particularly if those varieties introduce unusual characteristics.

\textbf{Varietais}: Common (e.g., Winesap, McIntosh, Golden Delicious, Braeburn, Jonathan), multi-use (e.g., Northern Spy, some Russets, Baldwin), any suitable wildings

\textbf{Estatísticas}: OG: 1.045 - 1.065 \\
\phantom{ } \hspace{16.5mm} FG: 0.995 - 1.020 \\
\phantom{ } \hspace{16.5mm} ABV: 4.5 - 8\%

\textbf{Exemplos Comerciais}: Æppeltreow Barn Swallow Cider, Bellwether Liberty Spy, Doc’s Hard Apple Cider, Seattle Cider Dry, Tandem Ciders Smackintosh, 2 Towns BrightCider, Uncle John’s Apple Hard Cider