\phantomsection
\subsection*{C1B. Heirloom Cider}
\addcontentsline{toc}{subsection}{C1B. Heirloom Cider}

\textit{\textbf{Heirloom Cider} is a broadly-defined style that often uses at least some cider apples to create a product having more tannin than Common Cider. It is usually made outside the regions associated with English, French, and Spanish Cider styles, and lacks the distinguishing MLF or rustic characteristics of those styles. It is a type of "craft" cider produced in North America, eastern England, and elsewhere in the world.}

\textbf{Impressões Gerais}: Combining the apple character and acidity of a Common Cider with the tannin of an English or French Cider, while retaining a clean fermentation profile.

\textbf{Aroma e Sabor}: The intensity of apple character, esters, and sweetness typically varies with the sweetness level. Heirloom variety cider apples may bring their own unique, often rustic, qualities. Acidity can be moderate to high. Tannins can be medium-low to medium-high. Tannins may add to the impression of dryness in the finish, while contributing flavors that are reminiscent of wood, leather, or apple skins. Acidity and tannin together balance the sweetness and provide structure to the cider; they are both typically present, and do not have to be at equal levels. Has a clean fermentation profile without MLF derived phenol or barnyard character. Mousiness is a serious fault. Light yeast character acceptable.

\textbf{Aparência}: Slightly cloudy to brilliant. Color ranges from straw to deep gold. Red-fleshed apple varieties can produce ciders with a blush hue.

\textbf{Sensação na Boca}: Medium to full body, depending on tannin level. Any astringency and bitterness from tannin should be no more than moderate. Any level of carbonation.

\textbf{Comentários}: Probably most similar to English Cider, but without any MLF phenols or barnyard character, and having a higher acid balance. Sometimes called Heritage Cider or Traditional Cider. The name heirloom implies the use of older, not-widely-grown cider apple varieties, not that there is some added prestige, especially relative to Common Cider.

\textbf{Instruções para Inscrição}: Entrants \textbf{MUST} specify both carbonation and sweetness levels. Entrants \textbf{MAY} specify varieties of apples used; if specified, a varietal character will be expected.

\textbf{Varietais}: Multi-use varieties from Common Cider and many of the same bittersweet and bittersharp varieties used in English or French Ciders, or other heirloom or cider varieties, crabapples, hybrids, tannic wildings

\textbf{Estatísticas}: OG: 1.050 - 1.080 \\
\phantom{ } \hspace{16.5mm} FG: 0.995 - 1.020 \\
\phantom{ } \hspace{16.5mm} ABV: 6 - 9\%

\textbf{Exemplos Comerciais}: Eve's Cidery Autumn's Gold, Farnum Hill Extra Dry, Redbyrd Orchard Cloudsplitter, Sea Cider Flagship, Snowdrift Cliffbreaks Blend, Tandem Ciders Crabster, West County Cider Redfield