\phantomsection
\subsection*{C2C. Ice Cider}
\addcontentsline{toc}{subsection}{C2C. Ice Cider}

\textit{A cider fermented from juice concentrated either by freezing fruit before pressing or by freezing juice to remove water. Fermentation stops or is arrested before reaching dryness.}

\textbf{Aroma e Sabor}: Fruity, with a depth and complexity of apple flavor. Smooth, rich, sweet, and dessert wine-like but with a balancing acidity, like in a Sauternes or other high-quality dessert wine. Acidity must be high enough to prevent it from being cloying. Has a bright character when fresh. Age can bring a deeper complexity with a darker fruit and sugar character, but this should not seem strongly caramelized. Noticeable volatile acidity, typically perceived as acetone, is a fault.

\textbf{Aparência}: Brilliant. Color is deeper than a standard cider, in the range of gold to amber. Aged examples may show darker shades of color.

\textbf{Sensação na Boca}: Full body. May be tannic (astringent or bitter) but this is generally slight to moderate, although higher balanced levels are allowable. Can be warming but should not be hot.

\textbf{Comentários}: The character differs from Applewine in that the ice cider process increases not only sugar (and hence, potential alcohol) but also acidity and all fruit flavor components proportionately. Differs from Fire Cider in that it lacks deeply caramelized flavors, but has a higher acidity to balance the sweetness. No additives are permitted in this style; in particular, sweeteners may not be used to increase gravity. This style originated in Quebec in the 1990s.

\textbf{Instruções para Inscrição}: Entrants \textbf{MUST} specify starting gravity, final gravity or residual sugar, and alcohol level. Entrants \textbf{MUST} specify carbonation level.

\textbf{Varietais}: Usually North American classic table fruit such as McIntosh or Cortland

\textbf{Estatísticas}: OG: 1.130 - 1.180 \\
\phantom{ } \hspace{16.5mm} FG: 1.050 - 1.085 \\
\phantom{ } \hspace{16.5mm} ABV: 7 - 13\%

\textbf{Exemplos Comerciais}: Champlain Orchards Honeycrisp Ice Cider, Cidrerie St-Nicolas Glace Du Verger Iced Orchard Cider, Domaine Pinnacle Cidre de Glace, Eden Heirloom Blend Ice Cider, Eve's Cider Essence, Les Vergers de la Colline Le Glacé, Windfall Orchard Ice Cider