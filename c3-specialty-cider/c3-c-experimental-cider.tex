\phantomsection
\subsection*{C3C. Experimental Cider}
\addcontentsline{toc}{subsection}{C3C. Experimental Cider}

\textit{This is an open-ended, catch-all category for cider with other ingredients or processes that do not fit any of the previous cider styles in categories C1 through C3. It also may be used for any other type of historical or regional traditional cider not already described. If the cider fits a previous style description, then it is not an Experimental Cider.}

\textbf{Aroma e Sabor}: The cider character must always be present, and must fit with added ingredients or process effects. If a spirit barrel was used, the character of the spirit (rum, whiskey, etc.) may range from subtle (barely recognizable) to balanced and complementary (short of dominating and overwhelming the cider character). Overall balance and drinkability are the critical success factors for this style. The resulting cider should contain recognizable experimental components, and be pleasant to drink.

\textbf{Aparência}: Clear to brilliant, as appropriate for the base style. Color should be that of a standard cider unless other ingredients or processes are expected to contribute color.

\textbf{Sensação na Boca}: Reflects the base style, but may also show tannic, astringent, bitter, heavy body, or other characteristics as determined by declared ingredients or processes.

\textbf{Comentários}: Some examples fitting this category include:
\begin{itemize}[leftmargin=3mm]
\item Cider with added honey (unless used in New England Cider, or if honey is dominant in the balance, which should be entered as a M2A Cyser under the Mead Guidelines)
\item Cider with other sweeteners
\item Ciders with both spices and other (non-apple) fruit
\item Cider/beer hybrids (graff/graf, snakebite)
\item Cider with a wood or barrel character that is a significant part
of the flavor profile
\item Cider that otherwise meets existing guideline definitions,
except that it is noticeably outside listed style parameters
(e.g., strength, sweetness, carbonation)
\item Regional, traditional, or historical styles not in the guidelines.
\end{itemize}
Regardless of experimental nature, the resulting beverage must be recognizable as a cider. The description of the cider is critical information for judges, and should be sufficient to allow them to understand the concept. If special ingredients are declared, they should be perceived (exception: potential allergens do not need to be perceivable, but must be declared).
Experimental cider may exceed the typical Vital Statistics ranges for declared base styles, especially when based on concentrated styles (C2C or C2D).

\textbf{Instruções para Inscrição}: Entrants \textbf{MUST} specify the ingredients or processes that make the entry an experimental cider. Entrants \textbf{MUST} specify both carbonation and sweetness levels. Entrants \textbf{MAY} specify a base style, or provide a more detailed description of the concept.

\textbf{Varietais}: Any, depending on base cider

\textbf{Estatísticas}: OG: 1.045 - 1.100 \\
\phantom{ } \hspace{16.5mm} FG: 0.995 - 1.020 \\
\phantom{ } \hspace{16.5mm} ABV: 5 - 12\%

\textbf{Exemplos Comerciais}: Cidergeist Beezy, Domaine Dupoint Cidre Reserve, Finnriver Fire Barrel, Snowdrift Cornice, Tandem Ciders Bee's Dream, Uncle John's Blossom Blend, Uncle John's Sidra de Tepache