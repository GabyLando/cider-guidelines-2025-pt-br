\phantomsection
\subsection*{C4A. Common Perry}
\addcontentsline{toc}{subsection}{C4A. Common Perry}

\textit{\textbf{Common Perry} is made from culinary (table) pears.}

\textbf{Impressões Gerais}: Mildly fruity, fuller-bodied. Usually semi-dry to semi-sweet. Still to pétillant, typically. Only very slight acetification is acceptable.

\textbf{Aroma e Sabor}: There is a fruity pear character, which can be mild but increases in sweeter examples. The pear character reflects flavors expected of fermented table pears, which may not taste strongly like fresh pears. Drier versions tend toward a profile similar to a young white wine. The acidity level should be mild to balanced, not sharp. Tannins can be mild to balanced, but should not add significant bitterness. The balance of acid and tannins is variable, but is generally even to acid-forward. Should not be mousy, ropy, or oily.

\textbf{Aparência}: Slightly cloudy to clear. Generally quite pale, with a straw to gold color.

\textbf{Sensação na Boca}: Relatively full body. Low to moderate tannins apparent as astringency. Still to sparkling carbonation, although most are no more than medium.

\textbf{Comentários}: Compared to Heirloom Perry, Common Perry has less tannin, more of a table fruit character, and can have more acidity. Some table pears contain significant amounts of sorbitol, which may give a dry perry an impression of sweetness. The perception of sorbitol as sweet is highly variable from one person to the next. Hence, entrants should specify sweetness according to actual residual sugar amount, and judges must be aware that they might perceive more sweetness than how the perry was entered. Back-sweetening with raw pear juice to achieve a recognizable flavor profile can be found in commercial examples, but this is not necessarily authentic or expected in perry from areas with a long, continuous tradition. The name common implies lack of rarity, not lack of quality or class. Sometimes called New World Perry or Modern Perry.

\textbf{Instruções para Inscrição}: Entrants \textbf{MUST} specify both carbonation and sweetness levels.

\textbf{Varietais}: Bartlett, Kiefer, Comice, Conference

\textbf{Estatísticas}: OG: 1.050 - 1.060 \\
\phantom{ } \hspace{16.5mm} FG: 1.000 - 1.020 \\
\phantom{ } \hspace{16.5mm} ABV: 5 - 8\%

\textbf{Exemplos Comerciais}: Æppeltreow Perry, EdenVale Pear Cider, Seattle Cider Perry, Snowdrift Semi-Dry Perry, Twin Pines Hammer Bent Perry, Uncle John's Perry