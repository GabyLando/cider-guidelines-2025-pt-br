\phantomsection
\subsection*{C4B. Heirloom Perry}
\addcontentsline{toc}{subsection}{C4B. Heirloom Perry}

\textit{A traditional perry made from "perry pears" grown specifically for that purpose, rather than for eating or cooking. Many ofthese varieties are nearly inedible due to high tannins; some are also quite hard. Perry pears may contain substantial amounts of sorbitol, a non-fermentable, sweet-tasting sugar alcohol.Hence a perry can exhibit the impression of sweetness, yet be completely dry (no RS).}

\textbf{Impressões Gerais}: Tannic and somewhat fruity, with a fuller body. English examples tend to be drier than French examples, so the sweetness level is variable. English and French examples may be carbonated to higher levels.
Aroma and Flavor: There is a noticeable fermented pear character, which can be subtle to quite fruity. The pear character can be more complex than a Common Perry, and does not taste strongly of table pears. The impression often tends toward that of a young white wine. A slight tannic bitterness is possible. The acidity level should be balanced, not sharp, as typically more tannin is present than acidity. Sorbitol may contribute to the impression of sweetness. Should not be mousy, ropy, or oily. Perry can sometimes have a very low level of natural acetification, which is unrelated to contamination.

\textbf{Aroma e Sabor}: Slightly cloudy to clear. Generally quite pale, with a straw to gold color. Still to sparkling carbonation, although most are no more than medium.

\textbf{Aparência}: Relatively full body. Moderate to high tannin apparent as astringency. Sorbitol can provide a smooth and slick texture. Should not seem syrupy.

\textbf{Sensação na Boca}: Relatively full body. Moderate to high tannin apparent as astringency. Sorbitol can provide a smooth and slick texture. Should not seem syrupy.

\textbf{Comentários}: Compared to Common Perry, Heirloom Perry is more tannin-forward, may have some bitterness, and has a more complex pear flavor. Note that a dry perry may give an impression of sweetness due to sorbitol in the pears, and perception of sorbitol as sweet is highly variable from one person to another. Hence entrants should specify sweetness according to actual residual sugar amount, and judges must be aware that they might perceive more sweetness than how the perry was entered. Back-sweetening with raw pear juice to achieve a recognizable flavor profile can be found in some commercial examples, but this is not necessarily authentic or expected in perry from areas with a long, continuous tradition. Sometimes called Traditional Perry or Heritage Perry. The name heirloom implies the use of older, not-widely-grown perry pear varieties, not that there is some added prestige, especially relative to Common Perry.

\textbf{Instruções para Inscrição}: Entrants \textbf{MUST} specify both carbonation and sweetness levels.

\textbf{Varietais}: Butt, Gin, Brandy, Barland, Blakeney Red, Thorn, Moorcroft

\textbf{Estatísticas}: OG: 1.050 - 1.070 \\
\phantom{ } \hspace{16.5mm} FG: 1.000 - 1.020 \\
\phantom{ } \hspace{16.5mm} ABV: 4 - 9\%

\textbf{Exemplos Comerciais}: Æppeltreow Orchard Oriole Perry, Burrow Hill Perry, Christian Drouin Poiré, Dragon's Head Sparkling Perry, Eric Bordelet Poiré Authentique, EZ Orchards Poire, Hogan's Classic Perry (UK), Oliver's Classic Perry