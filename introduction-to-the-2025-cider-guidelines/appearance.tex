\phantomsection
\subsection*{Appearance}
\addcontentsline{toc}{subsection}{Appearance}

\begin{itemize}
\item Clarity may range from good to brilliant. The lack of sparkling clarity is not a fault, but visible particles are undesirable. In some styles, a rustic lack of brilliance is common. Perries are notoriously difficult to clear; as a result, a slight haze is not a fault. However, a sheen in either cider or perry often indicates the early stages of lactic contamination and is a distinct fault.
\item Carbonation can vary from completely still to soda-like (spumante). Little to no carbonation is termed \textbf{still}, but may give a slight tickle on the tongue – it does not have to be dead flat. Moderate carbonation is termed \textbf{pétillant}. Highly carbonated is termed \textbf{sparkling}. At the higher carbonation levels, the mousse (head) may be retained for a short time. However, gushing, foaming, and difficult-to-manage heads are faults.
\item A cider or perry without additional ingredients is often a pale color, typically straw to gold. Be aware that some red-fleshed apples such as Redfield will give a blush or rosé hue that should not be misinterpreted as coming from other fruit; when in doubt, check the declared apple varieties. Dull, brownish shades can be an indication of oxidation, although darker tones could come from using low acid apples, keeving, aging or fermenting on wood, using concentration processes, or other reasons. Do not automatically assume oxidation by color alone. Obviously, examples containing added ingredients usually reflect the color of those additions.
\end{itemize}