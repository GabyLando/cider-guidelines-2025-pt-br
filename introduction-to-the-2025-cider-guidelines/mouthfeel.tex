\phantomsection
\subsection*{Mouthfeel}
\addcontentsline{toc}{subsection}{Mouthfeel}
\begin{itemize}
\item In general, cider and perry have a body and fullness akin to a light wine. Most cider styles have much less body than that of most beer. Some perries will have fuller bodies due to sorbitol (an unfermentable sugar alcohol), which can add a perception of sweetness.
\item Highly sparkling ciders can seem Champagne-like. Still ciders may seem lacking to novices since carbonation livens the presentation. Properly declared still ciders should not be penalized for lack of carbonation.
\item Tannin can affect mouthfeel by adding body, adding bitterness, or by increasing the perceived dryness of the finish. Tannic styles can have a pleasantly astringent mouthfeel resembling a red wine. Wine descriptors such as drying, fuzzy, or grippy may apply. An impression of wood, leather, dried leaves, or apple skins may also be present, with accompanying flavor effects.
\end{itemize}