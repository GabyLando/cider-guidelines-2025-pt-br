\phantomsection
\subsection*{Aroma and Flavor}
\addcontentsline{toc}{subsection}{Aroma and Flavor}

\begin{itemize}
\item Ciders and perries do not necessarily present overtly fruity aromas or flavors — in the same way that wine does not taste like grape juice or beer does not smell like wort. Drier styles of cider can develop a character that is more complex but less fruity. Cider and perry should not taste like a cocktail of raw fruit juice, carbonated water, and alcohol – they should taste fermented.
\item Winemakers classify smells as aroma (those derived from the ingredients) or bouquet (those derived from the process of fermentation and aging). Cider judges may benefit from thinking similarly, understanding how the cidermaking process transforms the raw ingredients into the finished product.
\item A clean fermentation profile is desirable in most styles, but this does not necessarily mean the absence of \textbf{yeast character}. Yeast can add estery notes or may have a light sulfury freshness; these are not faults. Aging on yeast can contribute light nutty, toasty, or bready notes.
\item Some cider styles exhibit distinctly non-fruity qualities, such as the smoky ham undertones of a dry English cider. Some regional styles have a rustic character.
\item \textbf{Sweetness} (residual sugar, or \textbf{RS}) ranges from absolutely dry (no RS) up to as much as that of dessert wines (10\% or more RS). Approximate measurements of RS and final gravity (FG) for the levels of sweetness are:
  \begin{itemize}
  \item[o] \textbf{Dry}: below 0.4\% RS, FG less than 1.002. No perception of sweetness, but the perception does not need to be bone dry.
  \item[o] \textbf{Semi-dry}: 0.4-0.9\% RS, FG 1.002-1.004. There is a hint of sweetness but the perception is still primarily dry. Also known as medium-dry or off-dry.
  \item[o] \textbf{Medium}: 0.9-2.0\% RS, FG 1.004-1.009. Sweetness is now a notable component of the overall balance.
  \item[o] \textbf{Semi-sweet}: 2.0-4.0\% RS, FG 1.009-1.019. The perception is sweet but still refreshing. Also known as medium-sweet.
  \item[o] \textbf{Sweet}: above 4.0\% RS, FG over 1.019. Like a dessert wine. Must not be syrupy or cloying.
  \end{itemize}
These numbers are meant to assist in entry decisions and to normalize regional perception differences, not be used as a disqualifying factor by judges. When close to the boundary between sweetness levels, enter based on the overall impression and how well it matches the descriptions for these levels.
Be aware that other factors (acidity, tannin, alcohol, dryness, other ingredients, etc.) affect the perception of sweetness. Do not rely solely on RS levels.
When judging, arrange samples in order of increasing sweetness. Understand that sweetness can mask faults — be more attentive to this in sweeter ciders. Likewise, do not overly penalize dry ciders for minor faults that may only be more evident due to the lack of sweetness.
In sweeter examples, non-fruity components of taste — particularly acidity and tannin — must complement the sweetness, or they will seem cloying (syrupy, heavily sweet) or flabby (sweetness unbalanced by acidity).
\item \textbf{Acidity} is an essential element of balance giving a clean, lively, bright, juicy, refreshing impression without being puckering. Acidity (from malic acid, and in some cases, lactic or other acids) must not be confused with acetification (from ethyl acetate or acetic acid — vinegar). The acrid aroma and tingling taste of volatile acidity (acetification) is a fault in most styles.
\item \textbf{Tannin} supplies astringency, body, and sometimes bitterness, which contribute to balance, structure, and drinkability. Excessive bitterness from tannin is a fault, whether from process or from ingredients. Table fruit typically has low tannin levels.
\item Ciders may undergo a \textbf{malolactic fermentation} (MLF), which reduces acidity by converting sharp malic acid into softer, rounder, less-acidic lactic acid. The result should not be flabby or too soft – the cider must remain refreshing. Perries should not undergo MLF because acetification may result.MLF can produce clean flavors, but MLF of tannic ciderapples often produces ethyl-phenols with spicy, smoky,smoked meat, phenolic, barnyard, funky, leathery, orhorsey flavors. Do not expect most or all of thesedescriptors simultaneously. Restrained, balanced levelsare optional but desirable in some regional styles. MLF is often mis-perceived as Brettanomyces (Brett), since they share many common descriptors, but Brett contamination is a serious fault. A dominating funky barnyard character from Brett is undesirable.Judges should be attentive to the possible mousy fault (THP, tetrahydropyridine), which is more likely in a higher pH cider that has undergone MLF. (For judges unable to detect the mousy fault, an alkaline oral rinse may be needed to confirm.)
\end{itemize}