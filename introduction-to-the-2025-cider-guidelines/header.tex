\section*{Introduction To Cider And Perry Styles (Categories C1-C4)}
\addcontentsline{toc}{section}{Introduction To Cider And Perry Styles (Categories C1-C4)}
\textit{This preamble applies to all the cider and perry styles, except where explicitly superseded in individual style descriptions. It identifies common characteristics and descriptions for all types of these beverages, and should be used as a reference when entering or judging. For more detailed information on applying the styles in a judging session, look at Studying for the Cider Exam on the BJCP website (Exam \& Certification, Cider Judge Program).}
\textbf{Cider} is the fermented juice of crushed apples. \textbf{Perry} is a similar beverage made from pears. In the United States, a distinction is made between hard cider (fermented, alcoholic) and sweet cider (unfermented, non-alcoholic). Elsewhere in the world, cider refers to the fermented product. We use the latter definition within these guidelines.
\textit{There are four categories in these guidelines for cider and perry: Traditional Cider (Category C1), Strong Cider (Category C2), Specialty Cider (Category C3), and Perry (Category C4). See the preamble to each category for more detailed descriptions. As with beer, there is no requirement that competitions judge these categories separately – individual styles may be grouped for judging and award purposes. Do not attempt to infer any deeper meaning from the names or groupings, as none is intended.}